\chapter{Разработка спецификации требований}
\label{chap:spec}

\section{Выявление ролей и функций, диаграммы прецедентов}
В разрабатываемой системе были выделены следующие роли: 
водитель, логист, бухгалтер. 
От введения отдельной роли администратора было решено отказаться, 
поскольку для любых его задач значительно эффективнее будет работать
напрямую с базой данных, применяя специализированные инструменты.

На рисунках
~\ref{fig:usecase:driver},
~\ref{fig:usecase:logist}~и
~\ref{fig:usecase:accounter} 
представлены диаграммы прецедентов для выделенных ролей.
На рисунке~\ref{fig:usecase:all} представлена диаграмма прецедентов,
которые относятся сразу ко всем ролям.

В приложении~\ref{chap:states} представлены диаграммы потоков экранов.

\image[0.5]
    {diagrams/usecase/driver}
    {Диаграмма прецедентов водителя}
    {fig:usecase:driver}

\image
    {diagrams/usecase/logist}
    {Диаграмма прецедентов логиста}
    {fig:usecase:logist}

\image
    {diagrams/usecase/accounter}
    {Диаграмма прецедентов бухгалтера}
    {fig:usecase:accounter}

\image
    {diagrams/usecase/all}
    {Диаграмма прецедентов всех пользователей}
    {fig:usecase:all}



\section{Текстовое описание прецедентов и макеты интерфейса}

\subsection{Прецедент <<войти в систему>>}
\textbf{Роли:} все \par
\textbf{Цель сценария:} войти в систему \par
\textbf{Предусловия:} открыто <<Окно входа в программу>> (рис. \ref{}) \par
\textbf{Основной сценарий:} 
\begin{enumerate}
    \item Ввести имя пользователя;
    \item Ввести пароль;
    \item Нажать кнопку <<Войти>>;
    \item Система проверяет наличие пользователя в базе;
    \item Система проверяет, что введён правильный пароль;
    \item Пользователь существует и пароль введён верно, 
        поэтому система открывает <<Домашнее окно>>.
\end{enumerate} \par
\textbf{Постусловия:} открыто <<Домашнее окно>>, окно входа закрыто. \par
\textbf{Альтернативная последовательность (пользователь не существует):} \par
\begin{enumerate}
    \item Вся последовательность сохраняется. Однако указанный пользователь 
        отсутствует в базе.
\end{enumerate} \par
\textbf{Постусловия:} отображается сообщение об ошибке. \par
\textbf{Альтернативная последовательность (пароль не верен):} \par
\begin{enumerate}
    \item Вся последовательность сохраняется. Однако введён неверный пароль.
\end{enumerate} \par
\textbf{Постусловия:} отображается сообщение об ошибке. \par

\subsection{Прецедент <<сменить пароль>>}
\textbf{Роли:} все \par
\textbf{Цель сценария:} изменить пароль пользователя для входа в систему \par
\textbf{Предусловия:} открыто окно <<Профиль>> (рис. \ref{}) \par
\textbf{Основной сценарий:} 
\begin{enumerate}
    \item Ввести текущий пароль в поле <<Текущий пароль>>;
    \item Ввести новый пароль в поле <<Новый пароль>>;
    \item Ввести новый пароль в поле <<Повторите новый пароль>>;
    \item Нажать кнопку <<Изменить пароль>>;
    \item Система проверяет совпадение пароля в двух полях ввода;
    \item Пароли совпадают, поэтому выводится окно с запросом подтверждения;
    \item В появившемся окне (рис. 10) нажать кнопку <<Подтвердить>>.
\end{enumerate} \par
\textbf{Постусловия:} пароль пользователя для входа в систему изменён. \par
\textbf{Альтернативная последовательность (отказ от смены пароля):} \par
\begin{enumerate}
    \item Последовательность сохраняется до нажатия кнопки <<Подтвердить>>, 
        вместо которой нажата кнопка <<Отмена>>.
\end{enumerate} \par
\textbf{Постусловия:} тображается сообщение об ошибке. \par
\textbf{Альтернативная последовательность (пароли не совпадают):} \par
\begin{enumerate}
    \item Вся последовательность сохраняется. Однако пароли не совпадают.
\end{enumerate} \par
\textbf{Постусловия:} отображается сообщение об ошибке. \par

\subsection{Прецедент <<выйти из системы>>}
\textbf{Роли:} все \par
\textbf{Цель сценария:} выйти из системы \par
\textbf{Предусловия:} открыто окно <<Профиль>> (рис. \ref{}) \par
\textbf{Основной сценарий:} 
\begin{enumerate}
    \item Нажать кнопку <<Выйти из системы>>;
    \item В появившемся окне (рис. \ref{}) нажать кнопку <<Подтвердить>>.
\end{enumerate} \par
\textbf{Постусловия:} сеанс пользователя будет завершён, 
    ранее открытые окна закрыты, откроется <<Окно входа в программу>> \par
\textbf{Альтернативная последовательность (остаться в системе):} \par
\begin{enumerate}
    \item Последовательность сохраняется до нажатия кнопки <<Подтвердить>>, 
        вместо которой нажата кнопка <<Отмена>>.
\end{enumerate} \par
\textbf{Постусловия:} Возврат к домашнему окну \par

\subsection{Прецедент <<просмотреть зарплату водителя>>}
\textbf{Роли:} бухгалтер \par
\textbf{Цель сценария:} получить список завершённых перевозок водителя 
    с информацией о вознаграждении за них \par
\textbf{Предусловия:} открыта вкладка <<Водители>> 
    в домашнем окне (рис. \ref{}), список водителей не пуст \par
\textbf{Основной сценарий:} 
\begin{enumerate}
    \item Выбрать в списке нужного водителя;
    \item Нажать кнопку <<Перевозки>>;
    \item В появившемся окне (рис. \ref) выбрать один из 
    предусмотренных периодов либо задать желаемый период вручную.
\end{enumerate} \par
\textbf{Постусловия:} доступен к просмотру список завершённых 
    перевозок водителя с информацией о вознаграждении за них \par

\subsection{Прецедент <<просмотреть список пользователей>>}
\textbf{Роли:} бухгалтер \par
\textbf{Цель сценария:} получить список пользователей с заданной ролью \par 
\textbf{Предусловия:} открыто домашнее окно \par
\textbf{Основной сценарий:} 
\begin{enumerate}
    \item Переключиться на вкладку с требуемой ролью пользователей
\end{enumerate} \par
\textbf{Постусловия:} доступен к просмотру список пользователей
    с выбранной ролью, содержащий всю имеющуюся в базе данных информацию \par

\subsection{Прецедент <<назначить перевозку>>}
\textbf{Роли:} логист \par
\textbf{Цель сценария:} назначить новую перевозку \par
\textbf{Предусловия:} открыта вкладка <<Перевозки>> в домашнем окне \par
\textbf{Основной сценарий:} 
\begin{enumerate}
    \item Нажать кнопку <<Добавить>>;
    \item В появившемся окне (рис. \ref{}) выбрать нужного водителя 
        в выпадающем списке <<Водитель>>;
    \item Если нужен второй водитель, выбрать его
        в выпадающем списке <<Второй водитель>>;
    \item Выбрать маршрут в выпадающем списке <<Маршрут>>;
    \item Если нужно, ввести значения премий водителей за данную перевозку;
    \item Нажать кнопку <<Подтвердить>>.
\end{enumerate} \par
\textbf{Постусловия:} назначена новая перевозка \par
\textbf{Альтернативная последовательность 
    (возврат к списку перевозок без назначения перевозки):} \par
\begin{enumerate}
    \item Последовательность сохраняется до нажатия кнопки <<Подтвердить>>, 
        вместо которой нажата кнопка <<Отмена>>.
\end{enumerate} \par
\textbf{Постусловия:} возврат к домашнему окну, изменения не вносятся \par

\subsection{Прецедент <<отменить перевозку>>}
\textbf{Роли:} логист \par
\textbf{Цель сценария:} пометить перевозку отменённой \par
\textbf{Предусловия:} открыта вкладка <<Перевозки>> в домашнем окне, 
     список перевозок не пуст \par
\textbf{Основной сценарий:} 
\begin{enumerate}
    \item Выбрать в списке перевозку, которую необходимо отменить;
    \item Нажать кнопку <<Отменить>>;
    \item В появившемся окне нажать кнопку <<Подтвердить>>.
\end{enumerate} \par
\textbf{Постусловия:} перевозка помечена отменённой \par
\textbf{Альтернативная последовательность 
    (возврат к списку перевозок без отмены перевозки):} \par
\begin{enumerate}
    \item Последовательность сохраняется до нажатия кнопки <<Подтвердить>>, 
        вместо которой нажата кнопка <<Отмена>>.
\end{enumerate} \par
\textbf{Постусловия:} возврат к домашнему окну, изменения не вносятся \par

\subsection{Прецедент <<завершить перевозку>>}
\textbf{Роли:} логист \par
\textbf{Цель сценария:} пометить перевозку успешно завершённой \par
\textbf{Предусловия:} открыта вкладка <<Перевозки>> в домашнем окне, 
     список перевозок не пуст \par
\textbf{Основной сценарий:} 
\begin{enumerate}
    \item Выбрать в списке перевозку, которую необходимо завершить;
    \item Нажать кнопку <<Завершить>>;
    \item В появившемся окне нажать кнопку <<Подтвердить>>.
\end{enumerate} \par
\textbf{Постусловия:} перевозка помечена успешно завершённой \par
\textbf{Альтернативная последовательность 
    (возврат к списку перевозок без завершения перевозки):} \par
\begin{enumerate}
    \item Последовательность сохраняется до нажатия кнопки <<Подтвердить>>, 
        вместо которой нажата кнопка <<Отмена>>.
\end{enumerate} \par
\textbf{Постусловия:} возврат к домашнему окну, изменения не вносятся \par

\subsection{Прецедент <<добавить маршрут>>}
\textbf{Роли:} логист \par
\textbf{Цель сценария:} внести в базу новый маршрут \par
\textbf{Предусловия:} открыта вкладка <<Маршруты>> 
    в домашнем окне (рис. \ref{}) \par
\textbf{Основной сценарий:} 
\begin{enumerate}
    \item Нажать кнопку <<Добавить>>;
    \item В появившемся окне (рис. \ref{}) ввести данные маршрута;
    \item Нажать кнопку <<Подтвердить>>.
\end{enumerate} \par
\textbf{Постусловия:}  \par
\textbf{Альтернативная последовательность ():} \par
\begin{enumerate}
\end{enumerate} \par
\textbf{Постусловия:} в базу добавлен новый маршрут \par
\textbf{Альтернативная последовательность 
    (возврат к списку маршрутов без добавления нового маршрута):} \par
\begin{enumerate}
    \item Последовательность сохраняется до нажатия кнопки <<Подтвердить>>, 
        вместо которой нажата кнопка <<Отмена>>.
\end{enumerate} \par
\textbf{Постусловия:} возврат к домашнему окну, изменения не вносятся \par

\subsection{Прецедент <<удалить маршрут>>}
\textbf{Роли:} логист \par
\textbf{Цель сценария:} пометить маршрут в базе неиспользуемым \par
\textbf{Предусловия:} открыта вкладка <<Маршруты>> в домашнем окне,
    список маршрутов не пуст \par
\textbf{Основной сценарий:} 
\begin{enumerate}
    \item Нажать кнопку <<Удалить>>;
    \item В появившемся окне (рис. \ref{}) ввести данные маршрута;
    \item Нажать кнопку <<Подтвердить>>.
\end{enumerate} \par
\textbf{Постусловия:}  \par
\textbf{Альтернативная последовательность ():} \par
\begin{enumerate}
\end{enumerate} \par
\textbf{Постусловия:} маршрут в базе помечен, как более не используемый \par
\textbf{Альтернативная последовательность 
    (возврат к списку маршрутов без удаления маршрута):} \par
\begin{enumerate}
    \item Последовательность сохраняется до нажатия кнопки <<Подтвердить>>, 
        вместо которой нажата кнопка <<Отмена>>.
\end{enumerate} \par
\textbf{Постусловия:} возврат к домашнему окну, изменения не вносятся \par

\subsection{Прецедент <<посмотреть свои перевозки>>}
\textbf{Роли:} водитель \par
\textbf{Цель сценария:} просмотреть список перевозок и 
    зарплату за определённый период \par
\textbf{Предусловия:} открыта вкладка <<Перевозки>> 
    в домашнем окне (рис. \ref{}) \par
\textbf{Основной сценарий:} 
\begin{enumerate}
    \item В области редактирования периодов выбрать 
        один из предусмотренных периодов либо задать желаемый период вручную;
    \item Система проверяет валидность периода;
    \item Период валиден, поэтому система отображает список 
        завершённых в выбранный период перевозок с информацией 
        об оплате за них по отдельности и в сумме за все.
\end{enumerate} \par
\textbf{Постусловия:} доступен к просмотру список завершённых 
    в выбранный период перевозок с информацией об оплате за них \par
\textbf{Альтернативная последовательность (указан невалидный период):} \par
\begin{enumerate}
    \item Вся последовательность сохраняется. Однако период невалиден.
\end{enumerate} \par
\textbf{Постусловия:} отображается сообщение об ошибке \par


\section{Описание форматов данных}

