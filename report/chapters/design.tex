\chapter{Объектно-ориентированное проектирование}
\label{chap:design}


\section{Диаграммы пригодности и последовательности}

\newcommand*\addUsecase[4]{
    \subsection{Прецедент <<#2>>}
    На рисунке \ref{fig:robustness:#1} представлена диаграмма 
    пригодности для данного прецедента.
    На рисунке \ref{fig:sequence:#1} представлена диаграмма 
    последовательности для данного прецедента.
    \image[#3]
        {diagrams/robustness/#1}
        {Диаграмма пригодности для прецедента <<#2>>}
        {fig:robustness:#1}
    \image[#4]
        {diagrams/sequence/#1}
        {Диаграмма последовательности для прецедента <<#2>>}
        {fig:sequence:#1}
}
\addUsecase{login}{войти в систему}{1}{1}
\addUsecase{changepassword}{сменить пароль}{1}{1}
\addUsecase{addtransp}{назначить перевозку}{1}{0.85}
\addUsecase{listusers}{просмотреть список пользователей}{0.75}{0.5}
\addUsecase{getsalary}{просмотреть свои перевозки}{0.7}{0.7}


\section{ER-диаграмма}
На рисунке \ref{fig:er:db} представлена ER-диаграмма 
разрабатываемого приложения в нотации Мартина.
\image
    {diagrams/er/db}
    {ER-Диаграмма}
    {fig:er:db}


\section{Диаграмма классов}
На рисунке \ref{fig:class} представлена диаграмма 
классов уровня проектирования для разрабатываемого приложения.
\image
    {diagrams/class/class}
    {Диаграмма классов уровня проектирования}
    {fig:class}

