\usepackage[utf8]{inputenc}
\usepackage[T2A]{fontenc}
\usepackage[russian]{babel}
\usepackage{indentfirst}
\usepackage{titlesec}
\usepackage{titletoc}
\usepackage{graphicx}
\usepackage{tabularx}
\usepackage{array}
\usepackage{hhline}
\usepackage{listings}
\usepackage{lastpage}
\usepackage{pdfpages}
\usepackage{caption}
\usepackage{amsmath}
\usepackage{hyperref}
\usepackage{enumitem}
\sloppy

\usepackage[
  a4paper,
  left=25mm,
  right=10mm,
  top=15mm,
  bottom=30mm
]{geometry} 

\renewcommand{\rmdefault}{Tempora-TLF}

\KOMAoptions{
  toc=bibliography,
  toc=chapterentrywithdots
}

% Красная строка
\setlength\parindent{12.5mm}

% Чтобы старался предотвратить страницы с одной строкой
\clubpenalty=9999
\widowpenalty=9999

\dottedcontents{chapter}[1em]{}{1em}{5}
\dottedcontents{section}[2.5em]{}{1.75em}{5}
\dottedcontents{subsection}[4.75em]{}{3em}{5}

\titleformat{\chapter}{\normalsize \bfseries}{\thechapter}{1ex}{}
\titleformat{\section}{\normalsize \bfseries}{\thesection}{1ex}{}
\titleformat{\subsection}{\normalsize \bfseries}{\thesubsection}{1ex}{}

\titlespacing*{\chapter}{\parindent}{*2}{*2}
\titlespacing*{\section}{\parindent}{*2}{*2}
\titlespacing*{\subsection}{\parindent}{*2}{*2}

% Глава без номера и занесения в содержание
\newcommand\phchapter[1]{
  \chapter*{\centerline{\MakeUppercase{#1}}}
  \par
}

% Глава без номера, но с занесением в содержание
\newcommand\unchapter[1]{
  \phchapter{#1}
  \addcontentsline{toc}{chapter}{#1}
  \par
}

% Заголовок библиографии
\renewcaptionname{russian}{\bibname}{Список использованных источников}
\makeatletter
\renewcommand*\bib@heading{
  \unchapter{\bibname}
  \@mkboth{\MakeMarkcase{\bibname}}{\MakeMarkcase{\bibname}}}
\makeatother

% Заголовок оглавления
\renewcaptionname{russian}{\contentsname}{
  \normalfont \bfseries \centerline{\MakeUppercase{Содержание}}
}

\captionsetup[figure]{
  name=Рисунок, 
  labelsep=endash, 
  justification=centering,
  % figurewithin=none
}
\captionsetup[table]{
  singlelinecheck=false,
  name=Таблица, 
  labelsep=endash, 
  justification=justified,
  % tablewithin=none
}

\renewcommand{\thesubfigure}{\asbuk{subfigure}}
\newcommand\image[4][1]{
  \begin{figure}[h!]
    \centering
    \includegraphics[width=#1\textwidth]{#2}
    \caption{#3}
    \label{#4}
  \end{figure}
}

% Меняем поведение appendix, чтобы был другой заголовок у приложений
\renewcommand{\appendixname}{Приложение}
\makeatletter
\renewcommand{\appendix}{\par%
  \renewcommand{\thechapter}{\Asbuk{chapter}}
  \setcounter{chapter}{0}
  \setcounter{section}{0}
  \setcounter{subsection}{0}
  \setcounter{figure}{0}
  \dottedcontents{chapter}[6.5em]{}{6.5em}{5}
  \titleformat{\chapter}[display]
    {\normalfont \bfseries}
    {\centering \MakeUppercase{\appendixname\ \thechapter}}
    {4ex}{\bfseries \centering}
  \titlecontents{chapter}[0]{}
    {\appendixname\ \contentslabel{0}\hspace{1.2em}}
    {}
    {\titlerule*[5]{.}\contentspage}
  % \captionsetup[figure]{figurewithin=chapter}
}
\makeatother

% Сливаем с файлом, чтобы была рамка
\AddToShipoutPicture{
  \AtPageLowerLeft{
    % Почему-то \AtPageLowerLeft смещает картинку вправо на 2mm
    % А eskdx генерирует рамку смещённую вниз на 1mm
    \put(-2mm, 1mm){
      \includegraphics[page=\value{page}]{./frames/eskd.pdf}
    }
  }
}

% Убираем номера страниц, они есть на рамке
\renewcommand*\pagemark{}

% Меняем настройки для маркированных списков
\setlist[itemize,1]{label=-}
\setlist[itemize]{
  wide, labelindent=\parindent, 
  noitemsep, topsep=0pt, parsep=0pt, partopsep=0pt
}
% Меняем настройки для нумерованных списков
\setlist[enumerate]{
  wide, labelindent=\parindent, 
  noitemsep, topsep=0pt, parsep=0pt, partopsep=0pt
}
